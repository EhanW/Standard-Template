\documentclass{article}
\input{pakages}
\title{Standard Template}
%%%%%%%%%%%%%%%%%%%%%
% use authblk package 
%%%%%%%%%%%%%%%%%%%%%
\usepackage{authblk}
\author[1]{Author \footnote{Contributed equally.}}
\author[2]{Coauthor \protect\CoAuthorMark}
\affil[1]{University}
\affil[2]{Institute}
\affil[ ]{author@email1.com, coauthor@email2.com}

% \author{Author \footnote{Contributed equally.} \\University\\ author@email1.com \and Coauthor \protect\CoAuthorMark\\Institute\\ coauthor@email2.com}


%%%%%%%%%%%%%%%%%%%%%
% use titling package 
%%%%%%%%%%%%%%%%%%%%%
% \usepackage{titling}
% \author{%
%   Yihan Wang \footnote{Contributed equally.} \\
%   Academy of Mathematics and Systems Science,\\
%   Chinese Academy of Sciences\\
%   University of Chinese Academy of Sciences\\
%   Beijing 100190 \\
%   \texttt{yihanwang@amss.ac.cn} \\
%   \and
%   Co-Author \\
%   Affiliation \\
%   Address \\
%   \texttt{email} \\
% }

\date{}
\begin{document}

\maketitle


\section{Symbols, Fonts and Maths}
$\mathrm{123asdfsa}$
\iffalse
abc
\fi

\begin{comment}
    aaaaa
\end{comment}

\textcolor{red}{123}
\textdagger

\begin{align}
    \frac{123}{222}\label{equ:1}\\
    \nicefrac{123}{222}\nonumber\\
    \mathscr{A}, \mathds{N}  \\
    \pdv[order={3, 4}]{x}{y,z}, \odv[order=1]{x}{z}, \odif[order=2]{x,y}, \pdif{x}{f} \\
    \dagger 
\end{align}


\teal{aaa}

\begin{empheq}[box=\fbox]{equation}
  E = mc^2
\end{empheq}
\begin{empheq}[box=\colorbox{yellow}]{equation}
  F = ma
\end{empheq}







\section{Reference and Citation} \label{sec:refer-cite}

\subsection{Reference}
\paragraph{Inner reference.}
\cref{sec:refer-cite}, \Cref{sec:refer-cite}, \Cref{equ:1}, \Cref{tab:single,tab:side-by-side,tab:warp-table}, \Cref{fig:single}, \Cref{fig:multiple-images}, \Cref{fig:multiple-images-1}, \Cref{fig:multiple-images-4}, \Cref{alg:algorithm}, \Cref{code}, Code \ref{code}

\paragraph{Hyper reference.}
\hyperref[sec:refer-cite]{Introduction},
\href{mailto:someone@example.com}{someone@example.com},
\href{https://www.example.com}{Example email}

\subsection{Citation}
\cite{VaswaniSPUJGKP17},
\citep{VaswaniSPUJGKP17},
\citet{VaswaniSPUJGKP17},
\citealt{VaswaniSPUJGKP17}, \citealp{VaswaniSPUJGKP17},
\citealt*{VaswaniSPUJGKP17},
\citeauthor{VaswaniSPUJGKP17}


\subsection{Quote}
Here is an example of an inline quotation: \endquote{Stay hungry, stay foolish.}

\begin{displayquote}
This is a block quotation. It is often used for long quotations that are more than a few lines long.
\end{displayquote}

As someone said, \enquote{He replied, \enquote{This is an example of nested quotations.}}









\clearpage
\section{Images and Tables}

\begin{wrapfigure}[14]{R}{0.4\textwidth}
    \vspace{-10pt}
    \centering
    \includegraphics[width=\linewidth]{example-image}
    \vspace{-15pt}
    \caption{Wrap figure.}
    \label{fig:wrap-figure}
\end{wrapfigure}
\lipsum[1]
\begin{wraptable}[10]{L}{0.4\textwidth}
    \vspace{-12pt}
    \centering
    \caption{Wrap table.}
    \label{tab:warp-table}
    \vspace{-8pt}
    \resizebox{\linewidth}{!}{%
    \begin{tabular}{l|cc}
    \toprule
    \multirow{2}{*}{\textbf{Columns}} & \multicolumn{2}{c}{Type 1} \\
    & \texttt{Attr. 1}  & \textbf{Attr. 2}    \\
    \midrule
    Method 1    & 20.00  & - \\
    Method 2    & 12.34  & 0.1234 \\
    Method 3    & 12.34  & 0.1234 \\
    Method 4    & 12.34  & 0.1234 \\
    \bottomrule
    \end{tabular}%
    }
\end{wraptable}
\lipsum[2-3]




\begin{figure}[ht]
    \centering
    \includegraphics[width=0.5\linewidth]{example-image}
    \caption{Single image.}
    \label{fig:single}
\end{figure}

\begin{table}[ht]
    \centering
    \begin{tabular}{l|cc|cc}
    \toprule
    \multirow{2}{*}{\textbf{Columns}} & \multicolumn{2}{c|}{Type 1} & \multicolumn{2}{c}{Type 2} \\
    & \texttt{Attr. 1}  & \textbf{Attr. 2}   & \texttt{Attr. 1}  & \textbf{Attr. 2}  \\
    \midrule
    Method 1    & 20.00  & - & 60.00 & -     \\
    Method 2    & 12.34  & 0.1234 & 56.78 & 0.5678     \\
    Method 3    & 12.34  & 0.1234 & 56.78 & 0.5678     \\
    Method 4    & 12.34  & 0.1234 & 56.78 & 0.5678     \\
    \bottomrule
    \end{tabular}%
    \caption{Single table.}
    \label{tab:single}
\end{table}



\begin{figure}[ht]
    \centering
    \begin{minipage}{0.32\textwidth}
        \includegraphics[width=\linewidth]{example-image}
        \caption{One of multiple images.}
        \label{fig:multiple-images-1}
    \end{minipage}
    \hfill
    \begin{minipage}{0.32\textwidth}
        \includegraphics[width=\linewidth]{example-image}
        \caption{One of multiple images.}
        \label{fig:multiple-images-2}
    \end{minipage}
    \hfill
    \begin{minipage}{0.32\textwidth}
        \includegraphics[width=\linewidth]{example-image}
        \caption{One of multiple images.}
        \label{fig:multiple-images-3}
    \end{minipage}
\end{figure}

\begin{figure}[ht]
    \centering
    \begin{subfigure}{0.32\textwidth}
        \includegraphics[width=\linewidth]{example-image}
        \caption{One of multiple images.}
        \label{fig:multiple-images-4}
    \end{subfigure}
    \hfill
    \begin{subfigure}{0.32\textwidth}
        \includegraphics[width=\linewidth]{example-image}
        \caption{One of multiple images.}
        \label{fig:multiple-images-5}
    \end{subfigure}
    \hfill
    \begin{subfigure}{0.32\textwidth}
        \includegraphics[width=\linewidth]{example-image}
        \caption{One of multiple images.}
        \label{fig:multiple-images-6}
    \end{subfigure}
    \caption{Multiple images as sub-figures for reference.}
\end{figure}

\begin{figure}[ht]
    \centering
    \includegraphics[width=0.32\linewidth]{example-image}
    \hfill
    \includegraphics[width=0.32\linewidth]{example-image}
    \hfill
    \includegraphics[width=0.32\linewidth]{example-image}
    \caption{Multiple images in a figure.}
    \label{fig:multiple-images}
\end{figure}



\Cref{fig:side-by-side}, \Cref{tab:side-by-side}, \Cref{fig:wrap-figure}, \Cref{tab:warp-table}

\begin{figure}[ht]
\begin{minipage}{0.47\textwidth}
\caption{Image side-by-side.}
\label{fig:side-by-side}
\includegraphics[width=\linewidth]{example-image}
\end{minipage}
\hfill
\begin{minipage}{0.49\textwidth}
\captionof{table}{Table side-by-side.}
\label{tab:side-by-side}
\resizebox{\linewidth}{!}{%
\begin{tabular}{l|cc|cc}
\toprule
\multirow{2}{*}{\textbf{Columns}} & \multicolumn{2}{c|}{Type 1} & \multicolumn{2}{c}{Type 2} \\
& \texttt{Attr. 1}  & \textbf{Attr. 2}   & \texttt{Attr. 1}  & \textbf{Attr. 2}  \\
\midrule
Method 1    & 20.00  & - & 60.00 & -     \\
Method 2    & 12.34  & 0.1234 & 56.78 & 0.5678     \\
Method 3    & 12.34  & 0.1234 & 56.78 & 0.5678     \\
Method 4    & 12.34  & 0.1234 & 56.78 & 0.5678     \\
Method 5    & 12.34  & 0.1234 & 56.78 & 0.5678     \\
Method 6    & 12.34  & 0.1234 & 56.78 & 0.5678     \\
Method 7    & 12.34  & 0.1234 & 56.78 & 0.5678     \\
Method 8    & 12.34  & 0.1234 & 56.78 & 0.5678     \\
Method 9    & 12.34  & 0.1234 & 56.78 & 0.5678     \\
Method 10   & 12.34  & 0.1234 & 56.78 & 0.5678     \\
Method 11   & 12.34  & 0.1234 & 56.78 & 0.5678     \\
Method 12   & 12.34  & 0.1234 & 56.78 & 0.5678     \\
Method 13   & \textbf{10.00}  & \textbf{0.1000} & \textbf{50.00} & \textbf{0.5000}      \\
\bottomrule
\end{tabular}%
}
\end{minipage}
\hfill
% \vspace{5pt}

\end{figure}







\section{Algorithm and Code}


\begin{algorithm}[htb]
\caption{Algorithm Example}
\label{alg:algorithm}
\begin{algorithmic}
\Require Input of the algorithm
\Ensure Output of the algorithm

\State Do something

\For{$i=1,\cdots,N$}
    \For{$j = 1,\cdots, M$} \Comment{Comment here}
        \State Do something
    \EndFor
    \If {Condition}
        \State Do something
    \EndIf
\EndFor
\end{algorithmic}
\end{algorithm}



\begin{lstlisting}[style=mypython,caption={Python code.}, label={code}]
def hello_world():
    print("Hello, world!")
\end{lstlisting}




\clearpage
\printbibliography


\end{document}
