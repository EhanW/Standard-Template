\documentclass{article}
\input{pakages}


\usepackage{natbib}



\title{Standard Template}
\author[1,2]{Yihan Wang \footnote{Contributed equally.}}
\author[3]{Co-Author \protect\CoAuthorMark \thanks{12}}
\author[4]{Advisor \thanks{12} \footnote{asdf}}

\affil[1]{Academy of Mathematics and Systems Science, Chinese Academy of Sciences}
\affil[2]{University of Chinese Academy of Sciences}
\affil[3]{Some University}
\affil[4]{Some Institute}


\date{}
\begin{document}

\maketitle

\section{Introduction}\label{sec:intro}
$\mathrm{123asdfsa}$


\cref{sec:intro}, \Cref{sec:intro} 

\hyperref[sec:intro]{123}
\href{mailto:someone@example.com}{someone@example.com}
\href{https://www.example.com}{Example}



\iffalse
abc
\fi

\begin{comment}
    aaaaa
\end{comment}

\textcolor{red}{123}
\textdagger

\begin{align}
    \frac{123}{222}\label{equ:1}\\
    \nicefrac{123}{222}\nonumber\\
    \mathscr{A}, \mathds{N}  \\
    \pdv[order={3, 4}]{x}{y,z}, \odv[order=1]{x}{z}, \odif[order=2]{x,y}, \pdif{x}{f} \\
    \dagger 
\end{align}

\Cref{equ:1}

\teal{aaa}

\begin{empheq}[box=\fbox]{equation}
  E = mc^2
\end{empheq}
\begin{empheq}[box=\colorbox{yellow}]{equation}
  F = ma
\end{empheq}


\cite{Knuth97}, \citep{Lamport86}, \cite{Smith20}

\lipsum[1-2]
\begin{wrapfigure}[14]{R}{0.4\textwidth}
    \vspace{-10pt}
    \centering
    \includegraphics[width=\linewidth]{example-image}
    \vspace{-15pt}
    \caption{Wrap figure.}
    \label{fig:wrap-figure}
\end{wrapfigure}
\lipsum[4-5]
\begin{wraptable}[10]{L}{0.4\textwidth}
    \vspace{-12pt}
    \centering
    \caption{Wrap table.}
    \label{tab:warp-table}
    \vspace{-8pt}
    \resizebox{\linewidth}{!}{%
    \begin{tabular}{l|cc}
    \toprule
    \multirow{2}{*}{\textbf{Columns}} & \multicolumn{2}{c}{Type 1} \\
    & \texttt{Attr. 1}  & \textbf{Attr. 2}    \\
    \midrule
    Method 1    & 20.00  & - \\
    Method 2    & 12.34  & 0.1234 \\
    Method 3    & 12.34  & 0.1234 \\
    Method 4    & 12.34  & 0.1234 \\
    \bottomrule
    \end{tabular}%
    }
\end{wraptable}
\lipsum[6-8]


\begin{lstlisting}
def hello_world():
    print("Hello, world!")
\end{lstlisting}
\begin{python}
    def hello_world():
    print("Hello, world!")
\end{python}



\begin{figure}[ht]
    \centering
    \includegraphics[width=0.5\linewidth]{example-image}
    \caption{Single image.}
    \label{fig:single}
\end{figure}


\begin{figure}[ht]
    \centering
    \begin{minipage}{0.32\textwidth}
        \includegraphics[width=\linewidth]{example-image}
        \caption{One of multiple images.}
        \label{fig:multiple-images-1}
    \end{minipage}
    \hfill
    \begin{minipage}{0.32\textwidth}
        \includegraphics[width=\linewidth]{example-image}
        \caption{One of multiple images.}
        \label{fig:multiple-images-2}
    \end{minipage}
    \hfill
    \begin{minipage}{0.32\textwidth}
        \includegraphics[width=\linewidth]{example-image}
        \caption{One of multiple images.}
        \label{fig:multiple-images-3}
    \end{minipage}
\end{figure}

\begin{figure}[ht]
    \centering
    \begin{subfigure}{0.32\textwidth}
        \includegraphics[width=\linewidth]{example-image}
        \caption{One of multiple images.}
        \label{fig:multiple-images-4}
    \end{subfigure}
    \hfill
    \begin{subfigure}{0.32\textwidth}
        \includegraphics[width=\linewidth]{example-image}
        \caption{One of multiple images.}
        \label{fig:multiple-images-5}
    \end{subfigure}
    \hfill
    \begin{subfigure}{0.32\textwidth}
        \includegraphics[width=\linewidth]{example-image}
        \caption{One of multiple images.}
        \label{fig:multiple-images-6}
    \end{subfigure}
    \caption{Multiple images as sub-figures for reference.}
\end{figure}

\begin{figure}[ht]
    \centering
    \includegraphics[width=0.32\linewidth]{example-image}
    \hfill
    \includegraphics[width=0.32\linewidth]{example-image}
    \hfill
    \includegraphics[width=0.32\linewidth]{example-image}
    \caption{Multiple images in a figure.}
    \label{fig:multiple-images}
\end{figure}



\Cref{fig:side-by-side}, \Cref{tab:side-by-side}, \Cref{fig:wrap-figure}, \Cref{tab:warp-table}

\begin{figure}[ht]
\begin{minipage}{0.47\textwidth}
\caption{Image side-by-side.}
\label{fig:side-by-side}
\includegraphics[width=\linewidth]{example-image}
\end{minipage}
\hfill
\begin{minipage}{0.49\textwidth}
\captionof{table}{Table side-by-side.}
\label{tab:side-by-side}
\resizebox{\linewidth}{!}{%
\begin{tabular}{l|cc|cc}
\toprule
\multirow{2}{*}{\textbf{Columns}} & \multicolumn{2}{c|}{Type 1} & \multicolumn{2}{c}{Type 2} \\
& \texttt{Attr. 1}  & \textbf{Attr. 2}   & \texttt{Attr. 1}  & \textbf{Attr. 2}  \\
\midrule
Method 1    & 20.00  & - & 60.00 & -     \\
Method 2    & 12.34  & 0.1234 & 56.78 & 0.5678     \\
Method 3    & 12.34  & 0.1234 & 56.78 & 0.5678     \\
Method 4    & 12.34  & 0.1234 & 56.78 & 0.5678     \\
Method 5    & 12.34  & 0.1234 & 56.78 & 0.5678     \\
Method 6    & 12.34  & 0.1234 & 56.78 & 0.5678     \\
Method 7    & 12.34  & 0.1234 & 56.78 & 0.5678     \\
Method 8    & 12.34  & 0.1234 & 56.78 & 0.5678     \\
Method 9    & 12.34  & 0.1234 & 56.78 & 0.5678     \\
Method 10   & 12.34  & 0.1234 & 56.78 & 0.5678     \\
Method 11   & 12.34  & 0.1234 & 56.78 & 0.5678     \\
Method 12   & 12.34  & 0.1234 & 56.78 & 0.5678     \\
Method 13   & \textbf{10.00}  & \textbf{0.1000} & \textbf{50.00} & \textbf{0.5000}      \\
\bottomrule
\end{tabular}%
}
\end{minipage}
\hfill
% \vspace{5pt}

\end{figure}




\bibliographystyle{plainnat}
\bibliography{reference.bib}


\end{document}
